
%%%%%%%%%%%%%%%%%%%%%%%%%%%%%%%%%%%%%%%%%%%%%%%%%%

\documentclass{beamer}
% \usepackage{graphicx}
\usepackage{beamerthemesplit}
\usepackage{color}

% \usepackage{color}
% \input{preamble1.tex}

% End of package list and other added codes.

%%%%%%%%%%%%%%%%%%%%%%%%%%%%%%%%%%%%%%%%%%%%%%%%%%

% \titlegraphic{\includegraphics[width=\textwidth,height=.5\textheight]{someimage}}
% \titlegraphic{\includegraphics[scale =  0.05 ]{C1_Core_2CG_RGB.jpg}}

\title{Prediction Intervals for\\
    the Area Under the ROC Curve}



\author[Lee Morin, Queen's University]{Lee Morin}

\institute[Queen's University]
{
    Department of Economics \\
    Queen's University
    % Capital One
    % \includegraphics[scale =  0.05 ]{C1_Core_2CG_RGB.jpg}
}

% \logo{\includegraphics[scale =  0.025 ]{C1_Core_2CG_RGB.jpg}}

%\author{Lee Morin}
%\institute[Queen's University]
%{
%    Department of Economics \\
%    Queen's University
%}

\date{\today}


%%%%%%%%%%%%%%%%%%%%%%%%%%%%%%%%%%%%%%%%%%%%%%%%%%


\begin{document}


%%%%%%%%%%%%%%%%%%%%%%%%%%%%%%%%%%%%%%%%%%%%%%%%%%

\frame{\titlepage}
% \frametitle{Agenda}

\section[Outline]{}

\frame{\tableofcontents}


%%%%%%%%%%%%%%%%%%%%%%%%%%%%%%%%%%%%%%%%%%%%%%%%%%


\section{Introduction}


%%%%%%%%%%%%%%%%%%%%%%%%%%%%%%%%%%%%%%%%%%%%%%%%%%

\begin{frame}
\frametitle{Predicting Performance of Classification Models}

% Tagline goes here
\begin{itemize}
    \item \textbf{What:} Method for calculating a prediction interval for the Area Under the ROC curve (AUROC)
    \begin{itemize}
        \item Area under the Receiver Operating Characteristic (ROC) curve is a measure of quality of a signal for a message
        \item In predictive modeling, it is often used as a measure of performance of a classification model
    \end{itemize}
    \item \textbf{Why:} Characterize the likely range of model performance when model is used for prediction
    \begin{itemize}
        \item In practice, businesses will use model until:
        \begin{itemize}
            \item Performance (AUROC) degrades
            \item Population changes
        \end{itemize}
    \end{itemize}
    \item \textbf{How:} Measure the variation in AUROC in terms of the variation in the underlying distribution of predictive variables
    \begin{itemize}
        \item Not only from sampling variation from a fixed distribution
    \end{itemize}
\end{itemize}

\end{frame}


%%%%%%%%%%%%%%%%%%%%%%%%%%%%%%%%%%%%%%%%%%%%%%%%%%




















%%%%%%%%%%%%%%%%%%%%%%%%%%%%%%%%%%%%%%%%%%%%%%%%%%


\end{document}

%%%%%%%%%%%%%%%%%%%%%%%%%%%%%%%%%%%%%%%%%%%%%%%%%%


