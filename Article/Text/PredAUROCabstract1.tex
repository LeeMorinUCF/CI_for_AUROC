
%%%%%%%%%%%%%%%%%%%%%%%%%%%%%%%%%%%%%%%%%%%%%%%%%%%%%%%%%%%%%%%%%%%%%%%%%%%%%%%%%%%%%%%%%%%%%%%%%%%%%%%%%%%%%%
% Article:
% Prediction Intervals for the Area Under the ROC Curve
% Title page and Abstract
%%%%%%%%%%%%%%%%%%%%%%%%%%%%%%%%%%%%%%%%%%%%%%%%%%%%%%%%%%%%%%%%%%%%%%%%%%%%%%%%%%%%%%%%%%%%%%%%%%%%%%%%%%%%%%%



\begin{center}
\vspace{3.0in}


{\Large
% Bootstrap Inference for the \\
% \vspace{0.10in}
% Area Under the ROC Curve \\
Forecast Intervals for the Area Under the ROC Curve \\
% \vspace{0.10in}
with a Time-varying Population \\
}

\vspace{0.75in}


{\large{Lealand Morin\footnote{I am grateful for helpful feedback
% from Morten Nielsen, James MacKinnon,
% Bill McLausland
% and
from seminar participants at
% Queen's University. %, other Universities
the University of Central Florida. %, other Universities
% and annual meetings of the Canadian Economics Association.
% I also gratefully acknowledge SSHRC for their generous support of this research.
All errors are my own. } \\
Department of Economics \\
% Queen's University \\
% morinl@econ.queensu.ca \\
% www.econ.queensu.ca/faculty/morin/home
%
College of Business Administration \\
University of Central Florida \\
Lealand.Morin@business.ucf.edu \\
https://business.ucf.edu/person/lealand-morin/ }}




\vspace{0.5in}

\today

\vspace{0.50in}



{\large{Abstract}}  \\
\vspace{0.25in}

\end{center}

There are several approaches to quantify the variability of the area under the receiver operating characteristic curve (AUROC) to measure the predictive performance of a binary classification model.
%
These methods typically involve calculating the variance of the estimate and calculating a $z$-statistic, under specified distributional assumptions
% or following some distribution-free approaches.
to characterize sampling variability.
%
%
In contrast, this paper proposes an approach
% to impose the null hypothesis of a particular AUROC,
to characterize variability in the AUROC by allowing for additional variation in the sampling distribution.
% without the need for distributional assumptions. % The approach involves solving for the closest distribution.
%
This additional variation corresponds to changes in the population over time.
%
Then, the variation is quantified by measuring the distance between joint distributions of subsamples from that of the full sample.
% 
Under this approach, distance is measured as the Kullback-Leibler divergence between the full sample and other potential samples. 
%
The extreme values of the AUROC statistics from distributions with the specified value of divergence
provide an interval for forecasting the AUROC to be expected in future applications of the model.
%

%The variation in the predicted AUROC is characterized by allowing for a distance between the empirical distribution and distributions that may hold while the classification model is used out of sample.
%The prediction interval is calculated by solving for the values of the AUROC statistic that lie furthest from that of the observed sample but within a specified distance from the empirical distribution.
%The bounds of the interval are calculated by first
%% solving for the values that match
%solving for the distribution with
%% In essence, the procedure involves re-weighting the observations to
%% those of
%the closest data generating process
%% that satisfies the null hypothesis of a particular AUROC value.
%that produces a particular AUROC value.
%%
%Then the bounds are calculated as the upper and lower values of the AUROC that lie exactly the specified distance from the empirical distribution.
%
%
%In this context, distance is defined by using nonparametric test statistics for the difference between the distributions of the model predictions under each of the positive and negative sets of observations.
%% Two distance metrics are analyzed, including the Chi-square statistic and an information-based criterion.
%% Under either approach,
%The key element is a recursive algorithm for reallocation of sampling weights in the empirical distribution.
%The algorithm solves for a set of fixed points that characterize the optimal distribution,
%from which the desired AUROC values are calculated.
%%
%%
%%
%Simulation exercises show the prediction intervals to have higher coverage rates than competing approaches that allow only for sampling variation.

% The result provides two approaches to test the statistical significance of a difference between competing models.
% The first is achieved by inverting the distance metric, finding the AUROC values corresponding to quantiles of the chi-square distribution.
% This is made more flexible by considering simulation methods, in which a bootstrap technique is employed to achieve refinements on the performance of the testing approach.
% Test performance is compared by analyzing power curves and coverage rates of confidence intervals.


\vspace{0.10in}




KEY WORDS: ROC curve, AUC, AUROC, classification models, predictive modeling, Mann-Whitney statistic, Wilcoxon statistic, Kullback-Leibler divergence.


%%%%%%%%%%%%%%%%%%%%%%%%%%%%%%%%%%%%%%%%%%%%%%%%%%%%%%%%%%%%%%%%%%%%%%%%%%%%%%
%%%%%%%%%%%%%%%%%%%%%%%%%%%%%%%%%%%%%%%%%%%%%%%%%%%%%%%%%%%%%%%%%%%%%%%%%%%%%%


\vspace{0.10in}
% \pagebreak


%%%%%%%%%%%%%%%%%%%%%%%%%%%%%%%%%%%%%%%%%%%%%%%%%%%%%%%%%%%%%%%%%%%%%%%%%%%%%%%%%%%%%%%%%%%%%%%%%%%%%%%%%%%%%%
%%%%%%%%%%%%%%%%%%%%%%%%%%%%%%%%%%%%%%%%%%%%%%%%%%%%%%%%%%%%%%%%%%%%%%%%%%%%%%%%%%%%%%%%%%%%%%%%%%%%%%%%%%%%%%%
