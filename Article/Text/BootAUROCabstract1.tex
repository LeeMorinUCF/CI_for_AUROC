
%%%%%%%%%%%%%%%%%%%%%%%%%%%%%%%%%%%%%%%%%%%%%%%%%%%%%%%%%%%%%%%%%%%%%%%%%%%%%%%%%%%%%%%%%%%%%%%%%%%%%%%%%%%%%%
% Article:
% Prediction Intervals for the Area Under the ROC Curve
% Title page and Abstract
%%%%%%%%%%%%%%%%%%%%%%%%%%%%%%%%%%%%%%%%%%%%%%%%%%%%%%%%%%%%%%%%%%%%%%%%%%%%%%%%%%%%%%%%%%%%%%%%%%%%%%%%%%%%%%%



\begin{center}
\vspace{3.0in}


{\Large
% Bootstrap Inference for the \\
% \vspace{0.10in}
% Area Under the ROC Curve \\
Bootstrap Inference on the Area Under the ROC Curve \\
% \vspace{0.10in}
% with a Time-varying Population \\
}

\vspace{0.75in}


{\large{Lealand Morin\footnote{I am grateful for helpful feedback
% from Morten Nielsen, James MacKinnon,
% Bill McLausland
% and
from seminar participants at
% Queen's University. %, other Universities
the University of Central Florida and at the meeting of the 2018 Canadian Econometric Study Group in Ottawa. %, other Universities
% and annual meetings of the Canadian Economics Association.
% I also gratefully acknowledge SSHRC for their generous support of this research.
All errors are my own. } \\
Department of Economics \\
% Queen's University \\
% morinl@econ.queensu.ca \\
% www.econ.queensu.ca/faculty/morin/home
%
College of Business Administration \\
University of Central Florida \\
Lealand.Morin@business.ucf.edu \\
https://business.ucf.edu/person/lealand-morin/ }}




\vspace{0.5in}

\today

\vspace{0.50in}



{\large{Abstract}}  \\
\vspace{0.25in}

\end{center}

There are several approaches to quantify the variability of the area under the receiver operating characteristic curve (AUROC) to measure the predictive performance of a binary classification model.
%
These methods typically involve calculating the variance of the estimate and calculating a $z$-statistic, under specified distributional assumptions
% or following some distribution-free approaches.
to characterize sampling variability.
%

Bootsrap methods are available but they are limited to the pairs bootstrap, 
which does not impose the null hypothesis. 

%
In contrast, this paper proposes an approach
to impose the null hypothesis of a particular AUROC,
% to characterize variability in the AUROC 
by shifting the sampling distribution
toward a distribution that has the specified AUROC under the null hypothesis.
% without the need for distributional assumptions. % The approach involves solving for the closest distribution.
%

%
Under this approach, distance is measured as the Kullback-Leibler divergence between the full sample and other potential samples.
%
The extreme values of the AUROC statistics from distributions with the specified value of divergence
provide an interval for forecasting the AUROC to be expected in future applications of the model.
%

In essence, the procedure involves re-weighting the observations to
those of
the closest data generating process
that satisfies the null hypothesis of a particular AUROC value.


The key element is a recursive algorithm for reallocation of sampling weights in the empirical distribution.
The algorithm solves for a set of fixed points that characterize the optimal distribution,
from which the desired AUROC values are calculated.


Simulation exercises show the size and power are better than existing alternatives.



\vspace{0.10in}




KEY WORDS: Bootstrap, ROC curve, AUC, AUROC, classification models, predictive modeling, Mann-Whitney statistic, Wilcoxon statistic, Kullback-Leibler divergence.


%%%%%%%%%%%%%%%%%%%%%%%%%%%%%%%%%%%%%%%%%%%%%%%%%%%%%%%%%%%%%%%%%%%%%%%%%%%%%%
%%%%%%%%%%%%%%%%%%%%%%%%%%%%%%%%%%%%%%%%%%%%%%%%%%%%%%%%%%%%%%%%%%%%%%%%%%%%%%


\vspace{0.10in}
% \pagebreak


%%%%%%%%%%%%%%%%%%%%%%%%%%%%%%%%%%%%%%%%%%%%%%%%%%%%%%%%%%%%%%%%%%%%%%%%%%%%%%%%%%%%%%%%%%%%%%%%%%%%%%%%%%%%%%
%%%%%%%%%%%%%%%%%%%%%%%%%%%%%%%%%%%%%%%%%%%%%%%%%%%%%%%%%%%%%%%%%%%%%%%%%%%%%%%%%%%%%%%%%%%%%%%%%%%%%%%%%%%%%%%
