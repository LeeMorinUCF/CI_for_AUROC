
%%%%%%%%%%%%%%%%%%%%%%%%%%%%%%%%%%%%%%%%%%%%%%%%%%%%%%%%%%%%%%%%%%%%%%%%%%%%%%%%%%%%%%%%%%%%%%%%%%%%%%%%%%%%%%%


\begin{figure}[h!]

\begin{center}

    % \hline
    \caption{Rates of Coverage and Correct Forecasts ($A_L = 0.75, A_H = 0.80$)} \label{fig:Coverage1}

    % \hline
    \begin{tabular}{l c c }

    Method & Coverage Rate & Correct Forecast Rate \\

    \hline
    
%    [1] "Coverage rates for n = 1100, A_L = 0.750000, A_H = 0.800000"
%           method coverage  forecast
%    1:      Bi-normal   0.6805 0.5912520
%    2: DeLong et. al.   0.6845 0.5979860
%    3:      Bootstrap   0.6730 0.5889715
%    4:    Upper Bound   0.9665 0.8654290
%    5:       Forecast   0.9940 0.9451650
    
    
          Bi-normal &  0.6805 & 0.5912 \\
     DeLong et. al. &  0.6845 & 0.5979 \\
          Bootstrap &  0.6730 & 0.5889 \\
        Upper Bound &  0.9665 & 0.8654 \\
           Forecast &  0.9940 & 0.9451 \\

    \hline

    \end{tabular}

    % \hline

\end{center}

    \footnotesize

        \textbf{Rates of Coverage and Correct Forecasts:}
        Coverage rates are the proportion of confidence intervals that contain the true values of the AUROC.
        Correct forecast rates are the proportion of confidence intervals that contain the estimated values of the AUROC.
        Number of replications is $1,000$ for all models, with $399$ bootstrap replications for bootstrap confidence intervals.
        Data generating process is a two-state regime-switching model with a bi-normal classification model in each state,
        with true AUROC values of $0.75$ in the low state and $0.80$ in the high state.
        % Bi-normal models have standard deviations of $1/sqrt{2}$.
        % Positive distributions have


    % \hline



\end{figure}


%%%%%%%%%%%%%%%%%%%%%%%%%%%%%%%%%%%%%%%%%%%%%%%%%%%%%%%%%%%%%%%%%%%%%%%%%%%%%%%%%%%%%%%%%%%%%%%%%%%%%%%%%%%%%%%

